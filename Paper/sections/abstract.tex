\section*{Abstract}
From nearly non-existence in the early 2000s, data breaches have exploded in terms of number of events and cumulative loss. As personal information has become the {\it ore} resource, mined by an Internet economy, which is largely based on targeted advertisement, there shall be no surprise that criminals steal and exploit personal data in their own way, such as for identity frauds. As people and organizations increasingly suffer their consequences, data breaches also undermine long-term confidence on the Internet as a reliable communication means, as good place for trustworthy data storage, and as a place in which a decent level of privacy can be maintained.



We conduct an exhaustive statistical analysis of data breaches as a new and fast evolving extreme risk, which has appeared and developed along with the Internet over a short period of nearly 15 years. In particular, we study (i) the evolution of the nature of the tail distribution over this period, (ii) the dynamics of the cumulative number of events and loss, and (iii) the improvements of event reporting and the completion of the catalog over time. Our results suggest that data breach cumulative loss has accelerated much faster than the total population of Internet users.

Based on the thorough assessment of dynamics and regularities, and taking into account the limitations of our study, we conduct out-of-sample predictions, and thus, provide some alarming forecasts concerning the evolution of data breach risk in the near future (5-year horizon).

